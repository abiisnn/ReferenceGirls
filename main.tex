% run the command ' lualatex -shell-escape Reference.tex ' twice in the terminal to visualize table of contents
\documentclass[twoside]{article}
\usepackage[utf8]{inputenc}
\usepackage[english]{babel}
\usepackage{geometry}
\usepackage{multicol}
\usepackage{minted}
\usepackage{python}
\usepackage[hidelinks]{hyperref}
\usepackage{fancyhdr}
\usepackage{listings}
\usepackage{pdfpages}
\usepackage{needspace}
\usepackage{sectsty}
\usepackage{array}
\usepackage{multirow} 
\usepackage{longtable}
\usepackage{xcolor}
\usepackage{afterpage}
\usepackage{amssymb}
\usepackage{amsmath}
\usepackage[inline]{enumitem}


\newcommand{\wdir}[1]{/home/san/Algorithms/Reference/#1}
\geometry{letterpaper, portrait, left=0.5cm, right=0.5cm, top=1.8cm, bottom=1cm}

\sectionfont{\Huge\bfseries\sffamily}

\setminted{
    style=tango,
    breaklines=true
}

\setlength{\headsep}{0.5cm}
\setlength{\columnsep}{0.5cm}
\setlength{\columnseprule}{0.01cm}
\renewcommand{\columnseprulecolor}{\color{gray}}

\pagestyle{fancy}
\pagenumbering{arabic}
\fancyhead{}
\fancyfoot{}
\fancyhead[LO,RE]{\textsf{First, solve the problem. Then, write the code.}}
\fancyhead[LE,RO]{\textsf{\leftmark}}
\fancyfoot[LE,RO]{\textbf{\textsf{\thepage}}}
 
\renewcommand{\headrulewidth}{0.01cm}
\renewcommand{\footrulewidth}{0.01cm}

\setlength{\parindent}{0em}
% column space
\setlength{\tabcolsep}{10pt} % Default value: 6pt
% upper and lower padding
\renewcommand{\arraystretch}{1.5} % Default value: 1

\definecolor{prussianblue}{rgb}{0.0, 0.19, 0.33}
\definecolor{indigo(dye)}{rgb}{0.0, 0.25, 0.42}
\definecolor{lapislazuli}{rgb}{0.15, 0.38, 0.61}
\definecolor{mediumelectricblue}{rgb}{0.01, 0.31, 0.59}
\definecolor{smalt(darkpowderblue)}{rgb}{0.0, 0.2, 0.6}
\definecolor{yaleblue}{rgb}{0.06, 0.3, 0.57}
\definecolor{skobeloff}{rgb}{0.0, 0.48, 0.45}
\definecolor{pinegreen}{rgb}{0.0, 0.47, 0.44}

\begin{document}
% \documentclass{article}
\usepackage{amsmath}
\usepackage{tikz}
\usepackage{epigraph}
\usepackage{lipsum}
\usepackage{geometry}

% The following code is borrowed from: https://tex.stackexchange.com/a/86310/10898
\definecolor{titlepagecolor}{cmyk}{1,.60,0,.40}
\newcommand\titlepagedecoration{%
\begin{tikzpicture}[remember picture,overlay,shorten >= -10pt]

\coordinate (aux1) at ([yshift=-15pt]current page.north east);
\coordinate (aux2) at ([yshift=-410pt]current page.north east);
\coordinate (aux3) at ([xshift=-4.5cm]current page.north east);
\coordinate (aux4) at ([yshift=-150pt]current page.north east);

\begin{scope}[titlepagecolor!40,line width=12pt,rounded corners=12pt]
\draw
  (aux1) -- coordinate (a)
  ++(225:5) --
  ++(-45:5.1) coordinate (b);
\draw[shorten <= -10pt]
  (aux3) --
  (a) --
  (aux1);
\draw[opacity=0.6,titlepagecolor,shorten <= -10pt]
  (b) --
  ++(225:2.2) --
  ++(-45:2.2);
\end{scope}
\draw[titlepagecolor,line width=8pt,rounded corners=8pt,shorten <= -10pt]
  (aux4) --
  ++(225:0.8) --
  ++(-45:0.8);
\begin{scope}[titlepagecolor!70,line width=6pt,rounded corners=8pt]
\draw[shorten <= -10pt]
  (aux2) --
  ++(225:3) coordinate[pos=0.45] (c) --
  ++(-45:3.1);
\draw
  (aux2) --
  (c) --
  ++(135:2.5) --
  ++(45:2.5) --
  ++(-45:2.5) coordinate[pos=0.3] (d);   
\draw 
  (d) -- +(45:1);
\end{scope}
\end{tikzpicture}%
}

\geometry{letterpaper, portrait, left=1cm, right=1cm, top=4.5cm, bottom=1cm}

\renewcommand{\epigraphrule}{1pt}
\renewcommand\epigraphsize{\normalsize}
\setlength\epigraphwidth{1\textwidth}

\DeclareFixedFont{\titlefont}{T1}{lmss}{b}{}{78pt}

\begin{document}
\begin{titlepage}
  \begin{center}
    \noindent
    \titlefont
    Competitive \\[70pt]
    Programming \\[70pt]
    Reference \\[70pt]

    \epigraph{
      \centering
      \fontfamily{lmss}
      \selectfont
      \Large
      First, solve the problem. Then, write the code.
    }
    {
      \centering
      \fontfamily{lmss}
      \selectfont
      \Large
      John Johnson
    }
    \null\vfill
    \fontfamily{lmss}
    \selectfont
    \large By \\[8pt]
    \Large Sergio Gabriel Sanchez Valencia \\[8pt]
    \Large gabrielsanv97@gmail.com \\[8pt]
    \Large searleser97
  \end{center}
  \titlepagedecoration
\end{titlepage}

\end{document}
\null
\thispagestyle{empty}
\newpage
\fontfamily{lmss}
\selectfont
\begin{multicols*}{2}
	\tableofcontents
	\newpage
	\cleardoublepage
	
	% ========================================
	%               TEMPLATE
	% ========================================
	\section{Template}
	
	% ========================================
	%               GRAPHS
	% ========================================
	\section{Graphs}
	    \subsection{DFS}
	    \subsection{BFS}
	    \subsection{Kruskal}
	    \subsection{Prim}
	    \subsection{Dijkstra}
	    \subsection{Union Find}
	
	% ========================================
	%               MATHS
	% ========================================
	\section{Maths}
	    \subsection{GCD}
	    \subsection{Criba}
	    
	% ========================================
	%            PROBABILITY
	% ========================================
	\section{}
	    
	% ========================================
	%             STRUCTURES
	% ========================================
	\section{Structures}
	    \subsection{Segment tree}
	    Este no tiene lazy
	    	\inputminted{cpp}{Estructuras/A.cpp}
    	\subsection{Sparse Table}
	    	\inputminted{cpp}{Estructuras/sparseTable.cpp}
    	\subsection{Fenwick Tree}
    	
\end{multicols*}
\end{document}